% !TEX TS-program = pdflatex
% !TEX encoding = UTF-8 Unicode

% This is a simple template for a LaTeX document using the "article" class.
% See "book", "report", "letter" for other types of document.

\documentclass[11pt]{scrartcl} % use larger type; default would be 10pt

\usepackage[utf8]{inputenc} % set input encoding (not needed with XeLaTeX)
\usepackage[title,titletoc,header]{appendix}
\usepackage{multicol}
% \usepackage{titling}
\usepackage{longtable}

\usepackage{amsmath}
\usepackage{hyperref}

%%% Examples of Article customizations
% These packages are optional, depending whether you want the features they provide.
% See the LaTeX Companion or other references for full information.

%%% PAGE DIMENSIONS
\usepackage[a4paper]{geometry} % to change the page dimensions
% \geometry{a4paper, margin=1.3in} % or letterpaper (US) or a5paper or....
% \geometry{margin=2in} % for example, change the margins to 2 inches all round
% \geometry{landscape} % set up the page for landscape
%   read geometry.pdf for detailed page layout information

\usepackage{graphicx} % support the \includegraphics command and options

% \usepackage[parfill]{parskip} % Activate to begin paragraphs with an empty line rather than an indent

%%% PACKAGES
\usepackage{booktabs} % for much better looking tables
\usepackage{array} % for better arrays (eg matrices) in maths
\usepackage{paralist} % very flexible & customisable lists (eg. enumerate/itemize, etc.)
\usepackage{verbatim} % adds environment for commenting out blocks of text & for better verbatim
\usepackage{subfig} % make it possible to include more than one captioned figure/table in a single float
% These packages are all incorporated in the memoir class to one degree or another...

\usepackage{pgfgantt} % gantt charts
% \usepackage[export]{adjustbox}[2011/08/13] % For centering wide figures

%%% HEADERS & FOOTERS
\usepackage{fancyhdr} % This should be set AFTER setting up the page geometry
\pagestyle{fancy} % options: empty , plain , fancy
\renewcommand{\headrulewidth}{0pt} % customise the layout...
\lhead{}\chead{}\rhead{}
\lfoot{}\cfoot{\thepage}\rfoot{}

%%% SECTION TITLE APPEARANCE
\usepackage{sectsty}
\allsectionsfont{\sffamily\mdseries\upshape} % (See the fntguide.pdf for font help)
% (This matches ConTeXt defaults)

%%% ToC (table of contents) APPEARANCE
\usepackage[nottoc,notlof,notlot]{tocbibind} % Put the bibliography in the ToC
\usepackage[titles,subfigure]{tocloft} % Alter the style of the Table of Contents
\renewcommand{\cftsecfont}{\rmfamily\mdseries\upshape}
\renewcommand{\cftsecpagefont}{\rmfamily\mdseries\upshape} % No bold!


\newcommand{\code}[1]{{\texttt{#1}}}
\newcommand{\codefile}[1]{{\textit{#1}}}
\newcommand{\program}[1]{\code{#1}}

%%% END Article customizations

% \setlength{\parindent}{0pt}
% \setlength{\parskip}{2ex plus 0.5ex minus 0.3ex}

%%% The "real" document content comes below...

% TODO: Catchy project title
\title{Computer Graphics Module 1 Update}
\subtitle{A 3D testing environment for an enhanced virtual reality system}
\author{Mitchell Metcalfe, James Ross-Gowan, Matthew Bray }

\date{\today} % Activate to display a given date or no date (if empty),
            % otherwise the current date is printed
\rhead{ COMP3320, Module 1, \today }

\begin{document}
% \newgeometry{top=2cm}
\maketitle
% \vspace{-1.5 cm}
% \tableofcontents
% \restoregeometry

\begin{abstract}

    This project aims to implement an ideal virtual environment for testing an
    enhanced virtual reality (VR) setup that utilises an Oculus Rift
    paired with an OptiTrack 3D motion capture system.
    The limited motion capture volume of this VR system requires a test scene
    that naturally limits the user to a small area.
    To meet this requirement, we propose to create an interactive 3D game that
    allows the player to explore the interior of a small spaceship. The
    spaceship will feature large windows, allowing the player to view a
    procedurally generated space visualisation outside the ship - maximising
    the user's sense of space and scale.

\end{abstract}

\section*{Project Schedule}

The following list outlines the project schedule. The program to achieve this
project schedule is presented in the Gantt chart in
Figure~\ref{gantt:schedule}.

    \begin{figure}[H]
        \makebox[\textwidth][c]{\resizebox{0.95\paperwidth}{!}{\begin{ganttchart}[
        hgrid,
        vgrid,
        title height = 1,
        x unit=0.3cm,
        y unit title=0.75cm,
        y unit chart=1cm,
        time slot format=isodate
        ] {2014-08-18}{2014-11-2}
    \gantttitlecalendar*{2014-08-18}{2014-10-31}{month=name} \\
    \gantttitlelist[title list options={var=\y, evaluate=\y as \x using "Week \y"}]{4,...,8}{7}
    \gantttitle{Mid-semester Break}{14}
    \gantttitlelist[title list options={var=\y, evaluate=\y as \x using "Week \y"}]{9,...,12}{7} \\

    % \ganttgroup{Group 1}{2014-08-20}{2014-10-5} \\
    % \ganttlinkedbar{Task 2}{2014-08-20}{2014-10-5} \ganttnewline
    
    \ganttmilestone{Proposal}{2014-08-20} \ganttnewline % 2014-08-20
    \ganttbar{Prepare Background Presenation}{2014-8-23}{2014-8-31} \ganttnewline
    \ganttmilestone{Presenation}{2014-8-31} \ganttnewline % 2014-09-01
    \ganttmilestone{Module 1}{2014-09-14} \ganttnewline % 2014-09-15
    \ganttmilestone{Module 2}{2014-10-12} \ganttnewline % 2014-10-13
    \ganttbar{Prepare Project Presenation}{2014-10-14}{2014-10-26} \ganttnewline % 
    \ganttmilestone{Completion}{2014-10-26} \ganttnewline % 2014-10-27
    
    % \ganttlink{elem2}{elem3}
    % \ganttlink{elem3}{elem4}
\end{ganttchart}
}}
        \caption[Project Schedule]{
            A Gantt chart illustrating the planned project schedule.
            Patterned grey bars represent optional tasks.
        }
        \label{gantt:schedule}
    \end{figure}
\end{document}

