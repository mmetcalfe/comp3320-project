\documentclass{beamer}
\usepackage{graphicx, epstopdf}
\usepackage{amsmath, amssymb, amsbsy, amstext}
% \usepackage{xcolor}

% \definecolor{darkgreen}{rgb}{0,0.6,0}
% \usepackage{pgf}
% \logo{\pgfputat{\pgfxy(-2,6)}{\pgfbox[center,base]{\includegraphics[height=2cm]{amsilogo.jpg}}}}

\usetheme{Dresden} %Dresden, Darmstadt, Warsaw
% \usecolortheme{dove}
\title[COMP3320 Background Presentation]{Global Illumination through Photon Mapping}
\subtitle{An overview}
\author{Mitchell Metcalfe, James Ross-Gowan,\\ Matthew Bray}
\institute{University of Newcastle}
\date{\today}

% \newenvironment{changemargin}[2]{% 
%   \begin{list}{}{% 
%     \setlength{\topsep}{0pt}% 
%     \setlength{\leftmargin}{#1}% 
%     \setlength{\rightmargin}{#2}% 
%     \setlength{\listparindent}{\parindent}% 
%     \setlength{\itemindent}{\parindent}% 
%     \setlength{\parsep}{\parskip}% 
%   }% 
%   \item[]}{\end{list}} 

\begin{document}
	\maketitle
	
	\section{Motivation}

		\subsection{The Rendering Equation}
			\begin{frame}{Single Integral}\end{frame}
			\begin{frame}{Split into terms}\end{frame}
			\begin{frame}{What does it mean?}\end{frame}

		\subsection{How does photography work}
			\begin{frame}{Light}\end{frame}
			\begin{frame}{Taking a photo}\end{frame}
			\begin{frame}{Why not just simulate that?}
				\begin{itemize}
					\item<2-> Lens is very small
					\item<3-> Many photons never reach it
				\end{itemize}
			\end{frame}

	\section{Photon mapping}
		\begin{frame}{Method}
			\begin{itemize}
				\item<2-> Cast many photons
				\item<3-> Simulate a few photon bounces
				\item<4-> Keep all of the resulting photon positions
			\end{itemize}
		\end{frame}
		
		\subsection{Photon simulation}
			\begin{frame}{Assumptions}
				\begin{itemize}
					\item<2-> All photons start with the same energy
					\item<3-> When a photon hits a surface, it has a probability of bouncing off
					\item<4-> Photons change based on the surfaces they hit
				\end{itemize}
			\end{frame}
			\begin{frame}{The Photons}
				Photon properties are:
				\begin{itemize}
					\item<2-> Colour
					\item<3-> Energy
					\item<4-> Direction
					\item<4-> Position
				\end{itemize}
			\end{frame}
			\begin{frame}{Illustration}\end{frame}

		\subsection{Rendering the scene}
			\begin{frame}{The photon map}
				\begin{itemize}
					\item<2-> Huge number of photons
					\item<3-> Invariant of camera position
					\item<4-> Still need to render the scene
				\end{itemize}
			\end{frame}
			\begin{frame}{The \(L_{indirect}\) term}
				Highlight the Lindirect term in the rendering equation
			\end{frame}
			\begin{frame}{Using the photon map}
				Image/Animation showing the `final gathering' step
			\end{frame}
			\begin{frame}{Representation}
				The photon map is huge and contains a complex distribution of photons.

				Use an acceleration structure:
				\begin{itemize}
					\item<2-> KD-tree
					\item<3-> Octree
					\item<4-> Spatial hashing
				\end{itemize}
			\end{frame}
			\begin{frame}{Other concerns}
				\begin{itemize}
					\item<2-> Very glossy surfaces
					\item<3-> Perfect reflectors
					\item<4-> Translucent objects
				\end{itemize}
				\pause Don't use the photon map. Just compute reflected rays (with importance sampling) as normal.
			\end{frame}
			\begin{frame}{Caustics}
				\begin{itemize}
					\item<2-> High photon density
					\item<3-> Low photon energy
				\end{itemize}

				Use a second photon map just for caustics and render it directly.
			\end{frame}

	\section{Discussion}
		\begin{frame}{Pros \& Cons}
				\begin{itemize}
					\item<2-> Traditionally not a realtime method
						\begin{itemize}
							\item<3-> Must generate photon map each frame for dynamic scenes
							\item<4-> GPU implementations do exist
						\end{itemize}
					\item<5-> Biased
						\begin{itemize}
							\item<6-> More photons increases image quality, but averaging many renders with a lower photon count does not converge to the same image
							\item<7-> Modifications exist to correct this
						\end{itemize}
				\end{itemize}
		\end{frame}

		\begin{frame}{Alternatives}
			\begin{itemize}
				\item<2-> Path Tracing
					\begin{itemize}
						\item<3-> Simple
						\item<4-> Unbiased
						\item<4-> Easier to parallelize
					\end{itemize}
				\item<5-> Radiosity
					\begin{itemize}
						\item<6-> Patch-based
						\item<7-> Point/Surfel based (Pixar's Renderman)
					\end{itemize}
				\item<5-> Ambient Occlusion
					\begin{itemize}
						\item<6-> A dirty hack
						\item<7-> Captures some important qualitative features of GI
					\end{itemize}
			\end{itemize}
		\end{frame}

		\subsection{Examples}

	\section{Ambient Occlusion}

\end{document}

