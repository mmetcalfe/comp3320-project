\documentclass{beamer}
\usepackage{graphicx, epstopdf}
\usepackage{amsmath, amssymb, amsbsy, amstext}
\usepackage{ upgreek }

\usepackage{pgf}
\usepackage{tikz}

%==================================
% Signed quote macro:
\def\signed #1{{\leavevmode\unskip\nobreak\hfil\penalty50\hskip2em
  \hbox{}\nobreak\hfil(#1)%
  \parfillskip=0pt \finalhyphendemerits=0 \endgraf}}

\newsavebox\mybox
\newenvironment{aquote}[1]
  {\savebox\mybox{#1}\begin{quote}}
  {\signed{\usebox\mybox}\end{quote}}
%==================================

% \newcommand{\slideframe}[2]{\only<#1>{\begin{center} \includegraphics[width=\textwidth]{images/#2.jpg} \end{center}}}
\newcommand{\slideframe}[2]{\only<#1>{\begin{center} \includegraphics[width=0.8\textwidth]{images/#2_adjusted.png} \end{center}}}

% \usepackage{xcolor}

% \definecolor{darkgreen}{rgb}{0,0.6,0}
% \usepackage{pgf}
% \logo{\pgfputat{\pgfxy(-2,6)}{\pgfbox[center,base]{\includegraphics[height=2cm]{amsilogo.jpg}}}}

\usetheme{Dresden} %Dresden, Darmstadt, Warsaw
% \usecolortheme{dove}
\title[COMP3320 Background Presentation]{Global Illumination through Photon Mapping}
\subtitle{An overview}
\author{Mitchell Metcalfe, James Ross-Gowan,\\ Matthew Bray}
\institute{University of Newcastle}
\date{\today}

% \newenvironment{changemargin}[2]{% 
%   \begin{list}{}{% 
%     \setlength{\topsep}{0pt}% 
%     \setlength{\leftmargin}{#1}% 
%     \setlength{\rightmargin}{#2}% 
%     \setlength{\listparindent}{\parindent}% 
%     \setlength{\itemindent}{\parindent}% 
%     \setlength{\parsep}{\parskip}% 
%   }% 
%   \item[]}{\end{list}} 

\begin{document}
  \maketitle
  
  \section{Motivation}

    \subsection{The Rendering Equation}

      \begin{frame}{Single integral}
        % From Wikipedia: http://en.wikipedia.org/wiki/Rendering_equation

        \[
          L_{\text{s}}(\mathbf x,\, \omega_{\text{r}}) \,=\,%
            L_e(\mathbf x,\, \omega_{\text{r}}) \ +\, %
            \int_\Omega %
              f_r(\mathbf x,\, \omega_{\text{i}},\, \omega_{\text{r}})\, %
                L_{\text{i}}(\mathbf x,\, \omega_{\text{i}})\, %
              (\omega_{\text{i}}\,\cdot\,\mathbf n)\,
            \operatorname d \omega_{\text{i}}
        \]

        \pause
        Emitted and reflected components:
        \[
          L_{\text{s}}(\mathbf x,\, \omega_{\text{r}}) \,=\,%
            L_e(\mathbf x,\, \omega_{\text{r}}) \ +\, %
            L_r(\mathbf x,\, \omega_{\text{r}})
        \]
      \end{frame}

      \begin{frame}{Split into terms}
        \begin{equation*}
        \begin{split}
          L_{\text{r}} \,=\,%
            &\int_\Omega %
              f_r\, %
                L_{i,l}\, %
              (\omega_{\text{i}}\,\cdot\,\mathbf n)\,
            \operatorname d \omega_{\text{i}}%
            \, +\\%
            &\int_\Omega %
              f_{r, s}\, %
                (L_{i,c} + L_{i,d})\, %
              (\omega_{\text{i}}\,\cdot\,\mathbf n)\,
            \operatorname d \omega_{\text{i}}
            \, +\\%
            &\int_\Omega %
              f_{r, d}\, %
                L_{i,c}\, %
              (\omega_{\text{i}}\,\cdot\,\mathbf n)\,
            \operatorname d \omega_{\text{i}}
            \, +\\%
            &\int_\Omega %
              f_{r, d}\, %
                L_{i,d}\, %
              (\omega_{\text{i}}\,\cdot\,\mathbf n)\,
            \operatorname d \omega_{\text{i}}
        \, \end{split}
        \end{equation*}
        where
        \[
          f_r = f_{r, s} + f_{r, d}%
          \, \text{ and } \, %
          L_i = L_{i,l} + L_{i,c} + L_{i,d}
        \]
      \end{frame}

      \begin{frame}{What does it mean?}
        \[
          L_r = L_{direct} + L_{specular} + L_{caustics} + L_{indirect}
        \]
      \end{frame}

    \subsection{How does Photography Work?}
      \begin{frame}{Light}
        \begin{aquote}{Wikipedia, \href{http://en.wikipedia.org/wiki/Light}{Light}}
          In common with all types of EMR, visible light is emitted and absorbed in \textbf{tiny `packets' called photons}, and exhibits properties of both waves and particles.
        \end{aquote}
        % - Wikipedia, http://en.wikipedia.org/wiki/Light
      \end{frame}

      \begin{frame}{Taking a photo}
        \slideframe{1}{image001}
        \slideframe{2}{image002}
        \slideframe{3}{image003}
        % \slideframe{4}{image004}
        % \slideframe{5}{image005}
        % \slideframe{6}{image006}
        \slideframe{4}{image007}
        \slideframe{5}{image008}
        \slideframe{6}{image009}
        \slideframe{7}{image010}
        \slideframe{8}{image011}
        % \slideframe{12}{image012}
        % \slideframe{13}{image013}
        \slideframe{9}{image014}
        \slideframe{10}{image015}
        \slideframe{11}{image016}
        \slideframe{12}{image017}
        \slideframe{13}{image018}
        \slideframe{14}{image019}
        \slideframe{15}{image020}
        \slideframe{16}{image021}
        \slideframe{17}{image022}
        \slideframe{18}{image023}
        \slideframe{19}{image024}
        \slideframe{20}{image025}
        \slideframe{21}{image026}
        \slideframe{22}{image027}
        \slideframe{23}{image028}
        \slideframe{24}{image029}
        \slideframe{25}{image030}
        \slideframe{26}{image031}
        % \slideframe{27}{image032}
        % \slideframe{28}{image033}
        % \slideframe{29}{image034}
      \end{frame}

      \begin{frame}{Why not just simulate that?}
        \begin{itemize}
          \item<2-> Lens is very small
          \item<3-> Many photons never reach it
        \end{itemize}
      \end{frame}

  \section{Photon Mapping}
    \begin{frame}{Method}
      \begin{itemize}
        \item<2-> Cast many photons
        \item<3-> Simulate a few photon bounces
        \item<4-> Keep all of the resulting photon positions
      \end{itemize}
    \end{frame}
    
    \subsection{Photon Simulation}
      \begin{frame}{Assumptions}
        \begin{itemize}
          \item<2-> All photons start with the same energy
          \item<3-> When a photon hits a surface, it has a probability of bouncing off
          \item<4-> Photons change based on the surfaces they hit
        \end{itemize}
      \end{frame}

      \begin{frame}{The photons}
        Photon properties are:
        \begin{itemize}
          \item<2-> Colour
          \item<3-> Power
          \item<4-> Position
          \item<5-> Direction
        \end{itemize}

        \slideframe{2}{image035}
        \slideframe{3}{image036}
        \slideframe{4}{image037}
        % \slideframe{5}{image038}
        \slideframe{5}{image039}
      \end{frame}

      \begin{frame}{Illustration}
        % \slideframe{1}{image040}
        \slideframe{1}{image041}
        \slideframe{2}{image042}
        \slideframe{3}{image043}
        \slideframe{4}{image044}
        \slideframe{5}{image045}
        \slideframe{6}{image046}
        \slideframe{7}{image047}
        \slideframe{8}{image048}
      \end{frame}

    \subsection{Rendering the Scene}
      \begin{frame}{The photon map}
        \begin{itemize}
          \item<2-> Huge number of photons
          \item<3-> Invariant of camera position
          \item<4-> Still need to render the scene
        \end{itemize}
      \end{frame}

      \begin{frame}{The \(L_{indirect}\) term}
        \[
          L_r = L_{direct} + L_{specular} + L_{caustics} + \color{red} L_{indirect}
        \]
      \end{frame}

      \begin{frame}{Using the photon map}
        % TODO: Image/Animation showing the `final gathering' step
        \slideframe{1}{image050}
        \slideframe{2}{image051}
        \slideframe{3}{image052}
        \slideframe{4}{image053}
        \slideframe{5}{image054}
        \slideframe{6}{image055}
        % \slideframe{8}{image056}
      \end{frame}

      \begin{frame}{Representation}
        The photon map is huge and contains a complex distribution of photons.

        Use an acceleration structure:
        \begin{itemize}
          \item<2-> KD-tree
          \item<3-> Octree
          \item<4-> Spatial hashing
        \end{itemize}
      \end{frame}

      \begin{frame}{Other concerns}
        \begin{itemize}
          \item<2-> Very glossy surfaces
          \item<3-> Perfect reflectors
          \item<4-> Translucent objects
        \end{itemize}
        \uncover<5->{Don't use the photon map. Just compute reflected rays (with importance sampling) as normal.}
      \end{frame}
      
      \begin{frame}{Caustics}
        \begin{center} \includegraphics[width=0.8\textwidth]{images/caustic.jpg} \end{center}
      \end{frame}
      
      \begin{frame}{Caustics}
        \[
          L_r = L_{direct} + L_{specular} + {\color{red} L_{caustics}} + L_{indirect}
        \]

        \begin{itemize}
          \item<2-> High photon density
          \item<3-> Low photon energy
        \end{itemize}

        \uncover<4->{Use a second photon map just for caustics and render it directly.}
        \uncover<5->{Only shoot caustics photons at highly specular/translucent objects.}
      \end{frame}

  \section{Discussion}
    \begin{frame}{Pros \& cons}
        \begin{itemize}
          \item<2-> Traditionally not a realtime method
            \begin{itemize}
              \item<3-> Must generate photon map each frame for dynamic scenes
              \item<4-> GPU implementations do exist
            \end{itemize}
          \item<5-> Biased
            \begin{itemize}
              \item<6-> Rendering with more photons increase image quality, but averaging many renders with a lower photon count does not converge to a correct image
              \item<7-> Modifications exist to correct this (e.g. Progressive Photon Mapping)
            \end{itemize}
        \end{itemize}
    \end{frame}

    \begin{frame}{Alternatives}
      \begin{itemize}
        \item<2-> Path Tracing
          \begin{itemize}
            \item<3-> Simple
            \item<4-> Unbiased
            \item<5-> Easier to parallelize
          \end{itemize}
        \item<6-> Radiosity
          \begin{itemize}
            \item<7-> Patch-based
            \item<8-> Point/Surfel based (Pixar's Renderman)
          \end{itemize}
        \item<9-> Ambient Occlusion
          \begin{itemize}
            \item<10-> A dirty hack
            \item<11-> Captures some important qualitative features of GI
          \end{itemize}
      \end{itemize}
    \end{frame}

    \subsection{Examples}

  \section{Ambient Occlusion}


\end{document}

