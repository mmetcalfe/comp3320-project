\documentclass{beamer}
\usepackage{graphicx, epstopdf}
\usepackage{amsmath, amssymb, amsbsy, amstext}
% \usepackage{xcolor}

% \definecolor{darkgreen}{rgb}{0,0.6,0}
% \usepackage{pgf}
% \logo{\pgfputat{\pgfxy(-2,6)}{\pgfbox[center,base]{\includegraphics[height=2cm]{amsilogo.jpg}}}}

\usetheme{Dresden} %Dresden, Darmstadt, Warsaw
% \usecolortheme{dove}
\title[COMP3320 Background Presentation]{Global Illumination through Photon Mapping}
\subtitle{An overview}
\author{Mitchell Metcalfe, James Ross-Gowan,\\ Matthew Bray}
\institute{University of Newcastle}
\date{\today}

% \newenvironment{changemargin}[2]{% 
%   \begin{list}{}{% 
%     \setlength{\topsep}{0pt}% 
%     \setlength{\leftmargin}{#1}% 
%     \setlength{\rightmargin}{#2}% 
%     \setlength{\listparindent}{\parindent}% 
%     \setlength{\itemindent}{\parindent}% 
%     \setlength{\parsep}{\parskip}% 
%   }% 
%   \item[]}{\end{list}} 

\begin{document}
	\maketitle
	
	\section{Motivation}

		\subsection{The Rendering Equation}
			\begin{frame}{Single Integral}\end{frame}
			\begin{frame}{Split into terms}\end{frame}
			\begin{frame}{What does it mean?}\end{frame}

		\subsection{How does photography work}
			\begin{frame}{Light}\end{frame}
			\begin{frame}{Taking a photo}\end{frame}
			\begin{frame}{Why not just simulate that?}
				\begin{itemize}
					\item<2-> Lens is very small
					\item<3-> Many photons never reach it
				\end{itemize}
			\end{frame}

	\section{Photon mapping}
		\begin{frame}{Method}
			\begin{itemize}
				\item<2-> Cast many photons
				\item<3-> Simulate a few photon bounces
				\item<4-> Keep all of the resulting photon positions
			\end{itemize}
		\end{frame}
		
		\subsection{Photon simulation}
			\begin{frame}{Assumptions}
				\begin{itemize}
					\item<2-> All photons start with the same energy
					\item<3-> When a photon hits a surface, it has a probability of bouncing off
					\item<4-> Photons change based on the surfaces they hit
				\end{itemize}
			\end{frame}
			\begin{frame}{The Photons}\end{frame}
			\begin{frame}{Illustration}\end{frame}

		\subsection{Rendering the scene}
			\begin{frame}{The photon map}\end{frame}
			\begin{frame}{The \(L_{indirect}\) term}\end{frame}
			\begin{frame}{Using the photon map}\end{frame}
			\begin{frame}{Other concerns}\end{frame}
			\begin{frame}{Representation}\end{frame}
			\begin{frame}{Caustics}\end{frame}

	\section{Discussion}
		\subsection{Pros \& Cons}
		\subsection{Examples}
		\subsection{Alternatives}

	\section{Ambient Occlusion}

\end{document}

