% !TEX TS-program = pdflatex
% !TEX encoding = UTF-8 Unicode

% This is a simple template for a LaTeX document using the "article" class.
% See "book", "report", "letter" for other types of document.

\documentclass[11pt]{scrartcl} % use larger type; default would be 10pt

\usepackage[utf8]{inputenc} % set input encoding (not needed with XeLaTeX)
\usepackage[title,titletoc,header]{appendix}
\usepackage{multicol}
% \usepackage{titling}
\usepackage{longtable}

\usepackage{amsmath}
\usepackage{hyperref}

%%% Examples of Article customizations
% These packages are optional, depending whether you want the features they provide.
% See the LaTeX Companion or other references for full information.

%%% PAGE DIMENSIONS
\usepackage[a4paper]{geometry} % to change the page dimensions
% \geometry{a4paper, margin=1.3in} % or letterpaper (US) or a5paper or....
% \geometry{margin=2in} % for example, change the margins to 2 inches all round
% \geometry{landscape} % set up the page for landscape
%   read geometry.pdf for detailed page layout information

\usepackage{graphicx} % support the \includegraphics command and options

% \usepackage[parfill]{parskip} % Activate to begin paragraphs with an empty line rather than an indent

%%% PACKAGES
\usepackage{booktabs} % for much better looking tables
\usepackage{array} % for better arrays (eg matrices) in maths
\usepackage{paralist} % very flexible & customisable lists (eg. enumerate/itemize, etc.)
\usepackage{verbatim} % adds environment for commenting out blocks of text & for better verbatim
\usepackage{subfig} % make it possible to include more than one captioned figure/table in a single float
% These packages are all incorporated in the memoir class to one degree or another...

\usepackage{pgfgantt} % gantt charts
% \usepackage[export]{adjustbox}[2011/08/13] % For centering wide figures

%%% HEADERS & FOOTERS
\usepackage{fancyhdr} % This should be set AFTER setting up the page geometry
\pagestyle{fancy} % options: empty , plain , fancy
\renewcommand{\headrulewidth}{0pt} % customise the layout...
\lhead{}\chead{}\rhead{}
\lfoot{}\cfoot{\thepage}\rfoot{}

%%% SECTION TITLE APPEARANCE
\usepackage{sectsty}
\allsectionsfont{\sffamily\mdseries\upshape} % (See the fntguide.pdf for font help)
% (This matches ConTeXt defaults)

%%% ToC (table of contents) APPEARANCE
\usepackage[nottoc,notlof,notlot]{tocbibind} % Put the bibliography in the ToC
\usepackage[titles,subfigure]{tocloft} % Alter the style of the Table of Contents
\renewcommand{\cftsecfont}{\rmfamily\mdseries\upshape}
\renewcommand{\cftsecpagefont}{\rmfamily\mdseries\upshape} % No bold!


\newcommand{\code}[1]{{\texttt{#1}}}
\newcommand{\libraryname}[1]{{\texttt{#1}}}
\newcommand{\codefile}[1]{{\textit{#1}}}
\newcommand{\program}[1]{\code{#1}}
\newcommand{\taskname}[1]{{\textit{#1}}}

%%% END Article customizations

% \setlength{\parindent}{0pt}
% \setlength{\parskip}{2ex plus 0.5ex minus 0.3ex}

%%% The "real" document content comes below...

% TODO: Catchy project title
\title{Graphics Project Final Report}
\subtitle{Game-engine demonstration via a dynamic space-themed environment}
% \subtitle{A 3D testing environment for an enhanced virtual reality system}
\author{ Mitchell Metcalfe, James Ross-Gowan }

\date{\today} % Activate to display a given date or no date (if empty),
            % otherwise the current date is printed
\rhead{ COMP3320, Final Report, \today }

\begin{document}
% \newgeometry{top=2cm}
\maketitle
% \vspace{-1.5 cm}
% \tableofcontents
% \restoregeometry

\begin{abstract}

    This report summarises the final result of our graphics project. We outline
    the initial goals and scope of the project, the changes that were made to
    the goals and scope throughout the project, and the final scope and focus
    of the project. Additionally, we discuss the features and effects
    implemented in our game-engine, and the  techniques used to achieve them.

\end{abstract}


\section{Project Goals}

    Our initial project proposal indicated that we would create a 3D engine
    using OpenGL and a test scene geared towards creating an immersive virtual
    reality system. Due to time constraints, changes in the structure of our
    team, and an unexpected lack of access to VR equipment, we chose to shift
    our focus to creating a 3D game engine that would support visually
    impressive graphics. The test scene we have created is similar to the scene
    described in our initial proposal, except that we have optimised for visual
    quality, rather than the high frame-rate and low latency that a VR
    application would demand.


\section{Final Submission}

    Our final project submission and demonstration consists of our OpenGL game
    engine code, and a space scene that has been constructed to showcase the
    many features of our game engine. The scene consists of a circular room
    suspended in space, with a large glass tube containing a moving, reflective
    sphere at its centre. The room has four curved, elliptic, glass windows
    equally spaced around its circumference. A complicated spaceship model
    (sourced from \url{http://www.cgtrader.com/free-3d-models/space/spacecraft-
    sci-fi/eagle-5-transport}) has been placed outside the room, opposite one
    of the windows. The room is illuminated by three spotlights, placed on the
    ceiling and inside the glass tube, and by sunlight from outside. The room
    contains a reflective robot that moves around the floor.
    A number of procedurally generated asteroids orbit the circular room.
    The scene has a background of stars and nebulae.

    % Image of the test scene

    The following subsection describes the features and techniques that were
    used to create the test scene.

    \subsection{Features Implemented}

        \subsubsection{Load 3D model files}

            The \libraryname{Assimp} library was used to load 3D model and
            material data from model files.

        \subsubsection{Textures}

            Our engine can render both textured and untextured meshes.

            The texture image filenames for textured materials are loaded by
            \libraryname{Assimp}, and then \libraryname{libjpg} or
            \libraryname{libpng} is used to load the image data before
            supplying it to OpenGL.

            Untextured meshes are supported by rendering the entire mesh in a
            single colour.

            A model can contain both textured and untextured meshes.

        \subsubsection{Bump mapping}

            Our engine supports bump map textures that are used to perturb
            normals during shading to simulate the interaction of lighting with
            fine surface details.

            A bump map is used to render the floor stubs
            on the floor of the spaceship in the demonstration scene.

        \subsubsection{Skybox rendering}

            A cube map is used to render a skybox. This gives the scene a
            plausible backdrop, increasing realism.

        \subsubsection{Environment mapped reflections}

            Each object can be assigned a cubemap texture to be used as an
            environment map for simulating reflections.

        \subsubsection{Filtered shadow maps}

            Shadow mapping is used to render shadows. Both spotlights and
            directional lights can use shadow maps. Spotlight shadows are not
            supported (as their implementation is more time-consuming).

            When determining the lighting contribution based on a shadow map,
            we sample the shadow map at multiple random points uniformly
            distributed around the actual shadow map coordinate for the point.
            The lighting contribution for the pixel is multiplied by the
            fraction of the sample points that were not in shadow. This has the
            effect of smoothing the edges of the shadows - removing the obvious
            jagged edges that result from limited shadow map resolution.

        \subsubsection{Forward and deferred rendering}

            To support multiple lights efficiently, we implemented a deferred
            renderer in addition to a forward renderer. When performing
            deferred rendering, we first render the eye-space normals, albedo,
            roughness, environment map reflection term, and depth to a set of
            framebuffers (collectively referred to as the `Geometry Buffer' or
            `G-Buffer'). These buffers are used as inputs to another shader
            which is used to add the contribution of each light to the screen
            in multiple additional render passes.

            The common optimisation of rendering `light shapes' to the stencil
            buffer before rendering lighting contributions was not implemented
            due to time constraints. As a result, several lights must be
            present in the scene before our deferred shading implementation
            exhibits a clear speed advantage.

        \subsubsection{Multiple lights}

            Both our forward and deferred shading implementations support an
            arbitrary number of lights in the scene. Multple lights are
            supported by simply rendering the contribution of each light to a
            temporary framebuffer before adding that framebuffer to the screen
            in turn.

        \subsection{Light types}

            Our engine supports three types of lights:

            \begin{itemize}

            \item[Point lights:]

                Simple point lights that emit light in all directions. These do
                not cast shadows.

            \item[Spotlights:]

                Spotlights that emit a cone of light in a given direction.
                These cast shadows and have an adjustable `inner angle', which
                allows the light emitted to be more intense at the centre of
                the cone, and fall off towards the edges.

            \item[Directional lights:]

                Defines a square region that emits light with a constant
                direction vector - as if a light source at infinity were
                shining through a square window. These cast shadows by using an
                orthographic projection to render the shadow map. A directional
                light is used to simulate sunlight in our demonstration scene.

            \end{itemize}

        \subsubsection{Dynamic reflection mapping}

            Our engine allows dynamic reflections on a per-object basis. If
            dynamic reflections are enabled for a given object, an environment
            map is rendered for that object centred at the object's origin.
            This environment map is then used to calculate the reflection term
            when shading the object.

            Environment maps are rendered using the forward shading
            implementation. All objects other than the given reflective object
            are rendered to the environment map.

            Rendering dynamic reflection maps is very slow, but significantly
            improves the realism of reflections.

        \subsubsection{Procedurally generated asteroids}

            The asteroids in the test scene are procedurally generated and
            placed.

            The mesh of each asteroid is generated by recursively subdividing
            the faces of an icosahedron, while randomly perturbing the distance
            from the icosahedron's center of each new point.

        \subsubsection{Phong lighting}

            The Phong lighting formula is used to calculate the lighting
            contribution from each light at each screen pixel. Each material
            has a configurable `shininess' parameter.

        \subsubsection{Fresnel effect for specular and environment reflections}

            An additional term is used in the lighting calculation to simulate
            the `Fresnel effect', where light appears more intense when
            reflected at glancing angles. This more accurately simulates
            lighting on many glossy surfaces.

        \subsubsection{Transparent materials}

            Transparency (or, more accurately, `dissolve') is supported as a
            material property. Transparent materials are only rendered using
            the deferred shading implementation. To render transparent
            materials, all fully opaque materials are first rendered to the
            g-buffer. Then, depth writes are disabled, and for each transparent
            mesh, the mesh is rendered to a framebuffer using the depth buffer
            from the g-buffer, before adding that framebuffer to the screen
            multiplied by the opacity of the material. The result is order-
            independent transparency that looks reasonable.


\section{Conclusion}

    In summary, we created an OpenGL game-engine that supports many features
    present in high-end game-engines. We demonstrated the features of our
    engine by rendering a complex, dynamic, space-themed scene in real-time.


\end{document}

