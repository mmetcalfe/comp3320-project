% !TEX TS-program = pdflatex
% !TEX encoding = UTF-8 Unicode

% This is a simple template for a LaTeX document using the "article" class.
% See "book", "report", "letter" for other types of document.

\documentclass[11pt]{scrartcl} % use larger type; default would be 10pt

\usepackage[utf8]{inputenc} % set input encoding (not needed with XeLaTeX)
\usepackage[title,titletoc,header]{appendix}
\usepackage{multicol}
% \usepackage{titling}
\usepackage{longtable}

\usepackage{amsmath}
\usepackage{hyperref}

%%% Examples of Article customizations
% These packages are optional, depending whether you want the features they provide.
% See the LaTeX Companion or other references for full information.

%%% PAGE DIMENSIONS
\usepackage[a4paper]{geometry} % to change the page dimensions
% \geometry{a4paper, margin=1.3in} % or letterpaper (US) or a5paper or....
% \geometry{margin=2in} % for example, change the margins to 2 inches all round
% \geometry{landscape} % set up the page for landscape
%   read geometry.pdf for detailed page layout information

\usepackage{graphicx} % support the \includegraphics command and options

% \usepackage[parfill]{parskip} % Activate to begin paragraphs with an empty line rather than an indent

%%% PACKAGES
\usepackage{booktabs} % for much better looking tables
\usepackage{array} % for better arrays (eg matrices) in maths
\usepackage{paralist} % very flexible & customisable lists (eg. enumerate/itemize, etc.)
\usepackage{verbatim} % adds environment for commenting out blocks of text & for better verbatim
\usepackage{subfig} % make it possible to include more than one captioned figure/table in a single float
% These packages are all incorporated in the memoir class to one degree or another...

\usepackage{pgfgantt} % gantt charts
\usepackage[export]{adjustbox}[2011/08/13] % For centering wide figures

%%% HEADERS & FOOTERS
\usepackage{fancyhdr} % This should be set AFTER setting up the page geometry
\pagestyle{fancy} % options: empty , plain , fancy
\renewcommand{\headrulewidth}{0pt} % customise the layout...
\lhead{}\chead{}\rhead{}
\lfoot{}\cfoot{\thepage}\rfoot{}

%%% SECTION TITLE APPEARANCE
\usepackage{sectsty}
\allsectionsfont{\sffamily\mdseries\upshape} % (See the fntguide.pdf for font help)
% (This matches ConTeXt defaults)

%%% ToC (table of contents) APPEARANCE
\usepackage[nottoc,notlof,notlot]{tocbibind} % Put the bibliography in the ToC
\usepackage[titles,subfigure]{tocloft} % Alter the style of the Table of Contents
\renewcommand{\cftsecfont}{\rmfamily\mdseries\upshape}
\renewcommand{\cftsecpagefont}{\rmfamily\mdseries\upshape} % No bold!


\newcommand{\code}[1]{{\texttt{#1}}}
\newcommand{\codefile}[1]{{\textit{#1}}}
\newcommand{\program}[1]{\code{#1}}

%%% END Article customizations

% \setlength{\parindent}{0pt}
% \setlength{\parskip}{2ex plus 0.5ex minus 0.3ex}

%%% The "real" document content comes below...

% TODO: Catchy project title
\title{Computer Graphics Project Proposal}
\subtitle{A 3D testing environment for an enhanced virtual reality system}
\author{Mitchell Metcalfe, James Ross-Gowan, Matthew Bray }

\date{\today} % Activate to display a given date or no date (if empty),
            % otherwise the current date is printed
\rhead{ COMP3320, Project Proposal, \today }

\begin{document}
% \newgeometry{top=2cm}
\maketitle
% \vspace{-1.5 cm}
% \tableofcontents
% \restoregeometry

\begin{abstract}

The Oculus Rift is a head-mounted virtual reality (VR) display developed by
Oculus VR. It contains a pair of LCD displays and lenses to provide stereoscopic
vision to the user, and also provides an IMU for tracking the orientation of
the user's head. Together, these allow an effective VR experience, provided
that the position of the user's head does not change. If a motion tracking
system were used instead of just the IMU, the users range of movement could be
increased to the entire motion tracking volume of the system. This would allow
for a significantly more realistic VR experience, provided that the 3D
environment naturally limited the user to the moion tracking volume. This
project aims to implement an ideal 3D environment for testing this enhanced VR
setup, by creating a realistic 3D environment that limits the user to a small area, while
maximising the user's sense of space and scale.

\end{abstract}

\section*{Project Description}
    


Oculus Rift \& Motion Tracking:
    Combine Oculus Rift with 3D motion tracking system to create a 3D virtual
    environment.

    Must be a small, enclosed environment, to discourage the player from
    leaving the motion tracking range, or walking into walls/obstacles in the
    room.

    A virtual spaceship interior could be ideal.

    Focus on creating a ideal 3D environment for testing the potential impact
    of the technology on the VR experience.

    Focus on creating an effective sense of space and scale, and avoiding
    distracting visual artifacts.

Spaceships:
    Player can move around the interior of a spaceship.
    Also, 3rd person flight controls.

    Planned visual effects:
        - Reflection mapping on some objects (possibly dynamic).
        - Shadow mapping + Soft Shadows
        - Space
        - Multiple lights + shadows
        - Screen-space ambient occlusion (time permitting)
        - Normal Mapping
        - Displacement mapping
        - Hardware tesselation

    Robot Theme:
        - There will be a robot on the spaceship.

    Required assets:
        - Spaceship model (interior and exterior).
        - Robot model
        - Space skybox
        - Planets/asteroids (procedurally generated?)


\section*{Team Description}
    Our team consists of the students listed in table \ref{table:teamMembers}.
    We have not yet planned specific roles or responsibilities for each team
    member.

    \begin{table}[h]
    \centering
        \begin{tabular}{@{}ll@{}}
        \toprule
            Name & Student Number \\ \midrule
            Mitchell Metcalfe & 3129636 \\
            James Ross-Gowan & 3102548 \\
            Matthew Bray & 3129671 \\ \bottomrule
        \end{tabular}
        \caption[Team members]{Names and student numbers of our project team's members.}
        \label{table:teamMembers}
    \end{table}

\section*{Project Schedule}

\begin{description}
    \item[Module 1:]
        \begin{itemize}
            \item Load and display textured 3D models
            \item Skybox rendering
            \item Static reflection mapping
            \item Per-pixel diffuse lighting (supporting a single light)
            \item NUClear based architecture
        \end{itemize}
    \item[Module 2:]
        \begin{itemize}
            \item Shadow mapping with soft shadows
            \item Dynamic reflection mapping on selected models
            \item Spaceship, planets,
        \end{itemize}
    \item[Final submission:]
        \begin{itemize}
            \item Add Oculus Rift support
            \item Integrate motion tracking system
        \end{itemize}
\end{description}

\begin{figure}
    \makebox[\textwidth][c]{\resizebox{0.95\paperwidth}{!}{\newcommand{\completedganttbar}[4][]{ %
    \ganttbar[bar/.append style={draw=gray, fill=gray},#1]{#2}{#3}{#4} %
}
\newcommand{\optionalganttbar}[4][]{ %
    \ganttbar[bar/.append style={draw=gray, pattern color=gray, pattern=north west lines},#1]{#2}{#3}{#4} %
}
\newcommand{\optionalganttlinkedbar}[4][]{ %
    \ganttlinkedbar[bar/.append style={draw=gray, pattern color=gray, pattern=north west lines},#1]{#2}{#3}{#4} %
}

\begin{ganttchart}[
        hgrid,
        vgrid={*6{black, dotted},*1{black, dashed}}, % Note: NO SPACES!
        title height = 1,
        x unit=0.3cm,
        y unit title=0.75cm,
        y unit chart=1cm,
        time slot format=isodate,
        % progress=today,
        % today=2014-8-20,
        % bar/.append style={fill=green},
        % bar incomplete/.append style={fill=white},
        % group incomplete/.append style={draw=black,fill=none},
        % progress label text={}
        ]
        {2014-08-4} % start date
        {2014-11-2} % end date
\setganttlinklabel{f-s}{}

% \gantttitlecalendar*{2014-08-18}{2014-10-31}{month=name} \\
\gantttitlecalendar*{2014-08-4}{2014-10-31}{month=name} \\
\gantttitlecalendar*{2014-08-4}{2014-9-21}{week=2}
\gantttitle{Mid-semester Break}{14}
\gantttitlecalendar*{2014-10-6}{2014-11-2}{week=9}

% \ganttlinkedbar{Task 2}{2014-08-20}{2014-10-5} \ganttnewline
    
\ganttnewline \ganttmilestone{Begin background research}{2014-08-3}
% \ganttnewline \ganttgroup{Group 1}{2014-08-20}{2014-10-5}

\ganttnewline \completedganttbar{Load and display textured 3D models}{2014-8-4}{2014-8-8}
\ganttnewline \completedganttbar{Skybox rendering}{2014-8-9}{2014-8-13}
\ganttnewline \completedganttbar{Static reflection mapping}{2014-8-14}{2014-8-17}
\ganttnewline \completedganttbar{Prepare project proposal}{2014-8-18}{2014-8-19}

\ganttnewline \ganttlinkedmilestone{Proposal Deadline}{2014-08-19} % 2014-08-20

\ganttnewline \completedganttbar{Prepare Background Presenation}{2014-8-20}{2014-8-31}

\ganttnewline \ganttlinkedmilestone{Background Presenation}{2014-8-31} % 2014-09-01

\ganttnewline \completedganttbar{Space skybox}{2014-9-1}{2014-9-8}
\ganttnewline \completedganttbar{Per-pixel diffuse lighting (supporting a single light)}{2014-09-4}{2014-09-8}
\ganttnewline \completedganttbar{Multiple lights}{2014-09-9}{2014-9-9}
\ganttnewline \completedganttbar{Shadow mapping}{2014-09-10}{2014-09-12}

\ganttnewline \ganttmilestone{Module 1 Deadline}{2014-09-14} % 2014-09-15

\ganttnewline \ganttbar{Spaceship model (interior and exterior)}{2014-09-11}{2014-10-12}
\ganttnewline \ganttbar{Procedurally generated asteroids and planets}{2014-9-10}{2014-9-19}
\ganttnewline \ganttbar{Robot model}{2014-09-20}{2014-10-12}
\ganttnewline
\ganttbar{Dynamic reflection mapping}{2014-09-12}{2014-9-12}
\ganttbar{}{2014-9-15}{2014-9-16}
\ganttnewline \ganttbar{Soft shadows}{2014-09-17}{2014-10-12}

\ganttnewline \ganttmilestone{Module 2 Deadline}{2014-10-12} % 2014-10-13

% \ganttnewline \optionalganttbar{Displacement maps}{2014-10-13}{2014-10-26}
% \ganttnewline \optionalganttbar{Normal Mapping}{2014-10-13}{2014-10-26}
% \ganttnewline \optionalganttbar{Bloom lighting effects}{2014-10-13}{2014-10-26}
\ganttnewline \optionalganttbar{Screen-space ambient occlusion}{2014-10-13}{2014-10-26}

% \ganttnewline \optionalganttbar{NUClear based architecture}{2014-10-13}{2014-9-28}
% \ganttnewline \optionalganttlinkedbar[link type=f-s]{Add Oculus Rift support}{2014-9-29}{2014-10-26}
\ganttnewline \optionalganttbar{Integrate motion tracking system}{2014-10-13}{2014-10-26}
\ganttnewline \optionalganttbar{Add Oculus Rift support}{2014-10-13}{2014-10-26}

\ganttnewline \ganttbar{Prepare Project Presenation}{2014-10-18}{2014-10-26} % 

\ganttnewline \ganttmilestone{Project Deadline}{2014-10-26} % 2014-10-27
    
    % \ganttlink{elem2}{elem3}
    % \ganttlink{elem3}{elem4}
\end{ganttchart}
}}
    \caption[Project Schedule]{A Gantt chart illustrating the planned project schedule.}
    \label{gantt:schedule}
\end{figure}
\end{document}

