% !TEX TS-program = pdflatex
% !TEX encoding = UTF-8 Unicode

% This is a simple template for a LaTeX document using the "article" class.
% See "book", "report", "letter" for other types of document.

\documentclass[11pt]{scrartcl} % use larger type; default would be 10pt

\usepackage[utf8]{inputenc} % set input encoding (not needed with XeLaTeX)
\usepackage[title,titletoc,header]{appendix}
\usepackage{multicol}
% \usepackage{titling}
\usepackage{longtable}

\usepackage{amsmath}
\usepackage{hyperref} 

%%% Examples of Article customizations
% These packages are optional, depending whether you want the features they provide.
% See the LaTeX Companion or other references for full information.

%%% PAGE DIMENSIONS
\usepackage[a4paper]{geometry} % to change the page dimensions
% \geometry{a4paper, margin=1.3in} % or letterpaper (US) or a5paper or....
% \geometry{margin=2in} % for example, change the margins to 2 inches all round
% \geometry{landscape} % set up the page for landscape
%   read geometry.pdf for detailed page layout information

\usepackage{graphicx} % support the \includegraphics command and options

% \usepackage[parfill]{parskip} % Activate to begin paragraphs with an empty line rather than an indent

%%% PACKAGES
\usepackage{booktabs} % for much better looking tables
\usepackage{array} % for better arrays (eg matrices) in maths
\usepackage{paralist} % very flexible & customisable lists (eg. enumerate/itemize, etc.)
\usepackage{verbatim} % adds environment for commenting out blocks of text & for better verbatim
\usepackage{subfig} % make it possible to include more than one captioned figure/table in a single float
% These packages are all incorporated in the memoir class to one degree or another...

%%% HEADERS & FOOTERS
\usepackage{fancyhdr} % This should be set AFTER setting up the page geometry
\pagestyle{fancy} % options: empty , plain , fancy
\renewcommand{\headrulewidth}{0pt} % customise the layout...
\lhead{}\chead{}\rhead{}
\lfoot{}\cfoot{\thepage}\rfoot{}

%%% SECTION TITLE APPEARANCE
\usepackage{sectsty}
\allsectionsfont{\sffamily\mdseries\upshape} % (See the fntguide.pdf for font help)
% (This matches ConTeXt defaults)

%%% ToC (table of contents) APPEARANCE
\usepackage[nottoc,notlof,notlot]{tocbibind} % Put the bibliography in the ToC
\usepackage[titles,subfigure]{tocloft} % Alter the style of the Table of Contents
\renewcommand{\cftsecfont}{\rmfamily\mdseries\upshape}
\renewcommand{\cftsecpagefont}{\rmfamily\mdseries\upshape} % No bold!


\newcommand{\code}[1]{{\texttt{#1}}}
\newcommand{\codefile}[1]{{\textit{#1}}}
\newcommand{\program}[1]{\code{#1}}

%%% END Article customizations

% \setlength{\parindent}{0pt}
% \setlength{\parskip}{2ex plus 0.5ex minus 0.3ex}

%%% The "real" document content comes below...

% TODO: Catchy project title
\title{Computer Graphics Project Proposal}
\author{Mitchell Metcalfe, James Ross-Gowan, Matthew Bray }

\date{\today} % Activate to display a given date or no date (if empty),
            % otherwise the current date is printed
\rhead{ COMP3320, Project Proposal, \today }

\begin{document}
% \newgeometry{top=2cm}
\maketitle
% \vspace{-1.5 cm}
% \tableofcontents
% \restoregeometry

\begin{abstract}

Oculus Rift & Motion Tracking:
    Combine Oculus Rift with 3D motion tracking system to create a 3D virtual
    environment.

    Must be a small, enclosed environment, to discourage the player from
    leaving the motion tracking range, or walking into walls/obstacles in the
    room.

    A virtual spaceship interior could be ideal.

Spaceships: 
    Player can move around the interior of a spaceship.
    Also, 3rd person flight controls.

    Planned visual effects:
        - Reflection mapping on some objects (possibly dynamic).
        - Shadow mapping + Soft Shadows
        - Space
        - Multiple lights + shadows
        % - Screen-space ambient occlusion (time permitting)

    Robot Theme:
        - There will be a robot on the spaceship.

    Required assets:
        - Spaceship model (interior and exterior).
        - Robot model
        - Space skybox
        - Planets/asteroids (procedurally generated?)

\end{abstract}

\section*{Project Description}


\section*{Team Description}
    \begin{description}
        \item[Mitchell Metcalfe, 3129636:]

        \item[James Ross-Gowan, 3102548:]

        \item[Matthew Bray, :]

    \end{description}
    
\section*{Project Schedule}

\begin{description}
    \item[Module 1:]
        - load and display textured 3d models
        - Skybox rendering
        - Static reflection mapping
        - handle single point lights
    \item[Module 2:]
        - 
    \item[Final submission:]
\end{description}

\end{document}

